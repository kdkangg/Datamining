
\section{Preprocessing and Mining}
\subsection{Data preprocessing}
Before applying classification algorithms to the data set in order to predict readiness scores for participants, we have completed a data preprocessing. The preprocessing part has contained the following steps:
\begin{enumerate}
    \item Since the explored data set consists of several tables for each participant, in the first step we have checked which tables are relevant for our task and which of them are not, then we have dropped irrelevant ones. Moreover, due to the incompleteness of the data for 12 and 13 participants, we have eliminated their data.
    \item In the second step, we have found out which attributes are appropriate for predicting the readiness scores and are contained in all participants' data. Therefore, we have dropped the following data: heart rate, exercise, reporting from google forms, injury and spre.
    \item In the next step, we have dealt with missing data. For some attributes we have replaced 0 values with NaN: calories, distance, lightly active minutes, resting heart rate, and steps (the reason: to explore such cases separately, because these attributes cannot be equal to 0); for the following attributes we have replaced missing values to the mean of a column: distance, resting heart rate, lightly active minutes, sedentary minutes, sleep composition score, sleep revitalization score, sleep duration score, deep sleep in minutes, sleep resting heart rate, sleep restlessness, sleep duration, sleep minutes asleep, sleep minutes awake, sleep time in bed, sleep efficiency and heart rate below zone1; we have dropped the rows with missing values and 0 values in a column readiness.
    \item In the fourth step, we have normalized the data set by using MinMaxScaler().
    \item The next step is splitting data into training and testing sets to apply KNN. We have split the original data set in a ratio of 75\% to 25\% for each participant, respectively. After this step, we have used KNN-method for training data and for filling the rest missing values in training and testing sets.
    \item In the sixth step, we have aggregated the data sets and added new columns which represent the days of the week in order to find out the impact of these values on readiness prediction scores.
    \item In the seventh step, we have split the data into training and testing sets according to the previous rule (75\% to 25\%). After that, we have aggregated the data from all participants.
    \item In the last step, we have scaled the following attributes - age and height across all participants and then used SMOTE() to deal with oversampling.
\end{enumerate}
\subsection{Data mining}
In order to predict the readiness scores for the participants which are in the range from 0 to 10, we have applied several classification methods.


\section{\textbf{3.Method}}
In order to find the optimal result for the dataset, we tried to build a highly accurate model for data analysis. We have tried more than six machine learning algorithms and in particular it is important to adjust the parameters of each algorithm. Optimising the parameters by manual selection would be time consuming and inefficient, especially if there were too many parameters [1][2]. We use the Grid Search, which is a classical hyperparametric optimisation method [3], with cross-validation. In general, the grid search is an exhaustive search based on a defined subset of the hyperparameter space. The hyperparameters are specified by the number of steps, the minimum value (lower bound), the maximum value (upper bound) and three different scales are available: logarithmic scale, linear scale and quadratic scale. Performance metrics exist to evaluate the performance of the different combinations [4]. However, when the data set is large, the processing time can be infinitely longer [5]. Hence, the ability to handle low-latitude data sets is superior to that of high-latitude data sets. But since the hyperparameters handled by the algorithm are usually independent of each other, grid search can be easily parallelized [3]. Cross-validation, applied in conjunction with the grid research, is a widely used strategy for model selection or estimator risk assessment and its one of the most popular methods of model and parameter selection in statistics and machine learning [6]. It is characterised by its simplicity and (apparent) generality and has many results in terms of the model selection performance of the cross-validation procedure [7].
Scikit-learn can evaluate the performance of the estimator and select parameters using cross-validation. Cross-validation assigns computations to several cores by wrapping the estimator in a GridSearchCV object,where the “CV” stands for “cross-validated” [8]. Most notably, GridSearchCV hardly falls into local optimisation and is compatible with almost every supervised learning algorithm, so we applied it to the six machine learning algorithms we chose. Using GridSearchCV, we found the optimal parameters for the voting algorithm [9].


\section{Reference}
[1] Friedrichs, F. and Igel, C., 2005. Evolutionary tuning of multiple SVM parameters. Neurocomputing, 64, pp.107-117.
[2] Rossi, A.L.D. and de Carvalho, A.C., 2008, October. Bio-inspired optimization techniques for svm parameter tuning. In 2008 10th Brazilian Symposium on Neural Networks (pp. 57-62). IEEE.
[3] Liashchynskyi, P. and Liashchynskyi, P., 2019. Grid search, random search, genetic algorithm: A big comparison for NAS. arXiv preprint arXiv:1912.06059.
[4] Syarif, I., Prugel-Bennett, A. and Wills, G., 2016. SVM parameter optimization using grid search and genetic algorithm to improve classification performance. Telkomnika, 14(4), p.1502.
[5] Huang, Q., Mao, J. and Liu, Y., 2012, November. An improved grid search algorithm of SVR parameters optimization. In 2012 IEEE 14th International Conference on Communication Technology (pp. 1022-1026). IEEE.
[6] Lei, J., 2020. Cross-validation with confidence. Journal of the American Statistical Association, 115(532), pp.1978-1997.
[7] Arlot, S. and Celisse, A., 2010. A survey of cross-validation procedures for model selection. Statistics surveys, 4, pp.40-79.
[8] Pedregosa, F., Varoquaux, G., Gramfort, A., Michel, V., Thirion, B., Grisel, O., Blondel, M., Prettenhofer, P., Weiss, R., Dubourg, V. and Vanderplas, J., 2011. Scikit-learn: Machine learning in Python. the Journal of machine Learning research, 12, pp.2825-2830.
[9] Zhao, S., Mao, X., Lin, H., Yin, H. and Xu, P., 2020. Machine Learning Prediction for 50 Anti-Cancer Food Molecules from 968 Anti-Cancer Drugs.




\section{KNN}
The k-nearest neighbor (kNN) method is one of the more popular classification methods in data mining and statistics, and its most notable features are its simplicity of implementation and remarkable classification performance[F1].
The model-free k-nearest neighbours (kNNs) method performs the classification task in two main steps. Firstly the calculation of the distance between the test samples and all the training samples is performed and ranked by ascending order. Secondly the labels of the selected nearest neighbours are assigned to the test samples according to the majority rule, cum kNN classification[F1].The common kNN method consists of two types, one specifies a fixed optimal k value (expert preset) to be applied to all test samples[F1][F2][F3][F4];The other specifies a different optimal k values to be applied to different test samples[F5][F6]. For instance, Sharma and Lall[F7] state that the fixed optimal k value for all test samples should be k =√n(where n > 100 and n is the number of training samples), while Zhu et al[F6] proposed to select different optimal k values for test samples by the tenfold cross-validation method. However, the traditional kNN method, which assigns a fixed value of K to all test samples, has proven to be less applicable in practice[F1]. Besides for classification tasks, the kNN method is further used for regression and imputation of missing data[F8]. kNN has two relatively significant drawbacks, the first being its relatively low operational efficiency – as a lazy learning method, it is banned in many applications, such as dynamic web mining for large repositories and secondly the choice of k value has an important impact on the results[F9].
In the preprocessing our group used the kNN imputation method designed according to Minkowski distance or its variants to handle missing data appropriately and to improve results of data analysis, generally this is a valid approach for numeric variables (features or attributes)[F10]. Under the Normalized Root Mean Square Error (RMSE) method kNN has shown the superiority over some other methods, for example, mean imputation, median imputation, predictive mean matching, Bayesian Linear Regression (norm), Linear Regression, non-Bayesian (norm.nob), and random sample[F11]. Our group chose to use the mean value per person to fill in the missing data. Then normalised the data. Generally speaking any type of problem is normalised in the pre-processing phase, especially in areas such as soft computing and cloud computing. The range of data is scaled according to the needs of the problem[F12].
kNN parameter selection was carried out using GridSearchCV. The parameters studied were the number of nearest neighbours, the metric of distance, weighting functions, and the leaf_size. The most popular parameter choices were evaluated, including nine k-values, three popular distance measures and two well-known weighting functions[F13].

\section{Reference}
[F1] Zhang, S., Li, X., Zong, M., Zhu, X. and Wang, R., 2017. Efficient kNN classification with different numbers of nearest neighbors. IEEE transactions on neural networks and learning systems, 29(5), pp.1774-1785.
[F2] Zhang, S., 2008. Parimputation: From imputation and null-imputation to partially imputation. IEEE Intell. Informatics Bull., 9(1), pp.32-38. 
[F3] Góra, G. and Wojna, A., 2002, August. RIONA: A classifier combining rule induction and k-NN method with automated selection of optimal neighbourhood. In European Conference on Machine Learning (pp. 111-123).
[F4] Springer, Berlin, Heidelberg. Li, B., Chen, Y.W. and Chen, Y.Q., 2008. The nearest neighbor algorithm of local probability centers. IEEE Transactions on Systems, Man, and Cybernetics, Part B (Cybernetics), 38(1), pp.141-154.
[F5] Wang, J., Neskovic, P. and Cooper, L.N., 2006. Neighborhood size selection in the k-nearest-neighbor rule using statistical confidence. Pattern Recognition, 39(3), pp.417-423. 
[F6] Zhu, X., Zhang, S., Jin, Z., Zhang, Z. and Xu, Z., 2010. Missing value estimation for mixed-attribute data sets. IEEE Transactions on Knowledge and Data Engineering, 23(1), pp.110-121.
[F7] Lall, U. and Sharma, A., 1996. A nearest neighbor bootstrap for resampling hydrologic time series. Water resources research, 32(3), pp.679-693.
[F8] Zhang, S., Li, X., Zong, M., Zhu, X. and Cheng, D., 2017. Learning k for knn classification. ACM Transactions on Intelligent Systems and Technology (TIST), 8(3), pp.1-19.
[F9] Guo, G., Wang, H., Bell, D., Bi, Y. and Greer, K., 2003, November. KNN model-based approach in classification. In OTM Confederated International Conferences" On the Move to Meaningful Internet Systems" (pp. 986-996). Springer, Berlin, Heidelberg.
[F10] Zhang, S., 2012. Nearest neighbor selection for iteratively kNN imputation. Journal of Systems and Software, 85(11), pp.2541-2552.
[F11] Jadhav, A., Pramod, D. and Ramanathan, K., 2019. Comparison of performance of data imputation methods for numeric dataset. Applied Artificial Intelligence, 33(10), pp.913-933.
[F12] Patro, S. and Sahu, K.K., 2015. Normalization: A preprocessing stage. arXiv preprint arXiv:1503.06462.
[F13] Batista, G.E.A.P.A. and Silva, D.F., 2009, August. How k-nearest neighbor parameters affect its performance. In Argentine symposium on artificial intelligence (pp. 1-12).

3.5 Decision Tree
3.5.1 approches
To solve the problem, we use the decision trees method.[Y1-2]. Decision tree (DT) induction [Y3] is a supervised classification and prediction technique with a long history going ba4ck over three decades [Y4]. A decision tree is a tree where each node shows a feature (attribute), each link (branch) shows a decision (rule) and each leaf shows an outcome (categorical or continues value)[Y5]Decision Tree algorithm as follows, Firstly, Set the dataset's best feature as the root of the tree. Secondly, Dataset is split into test and train sets. Subsets should be made in such a way that each subset contains information with the feature attribute like that. Then, On each subset, the steps above are repeated until we get leaves in the tree.[Y6]





3.5.3 parameter setting
We tried different combinations of the parameters ,the given following is the best choice. For parameters, use criterion, The function to measure the quality of a split. Supported criteria are “gini” for the Gini impurity and “log\_loss” and “entropy” both for the Shannon information gain.





References
[Y1] A. Albu, From logical inference to decision trees in medical diagnosis, Proceedings of 2017 EHealth and Bioengineering Conference (2017) 65-68. doi: 10.1109/EHB.2017.7995362 
[Y2] M.D.A. Praveena, J. S. Krupa, S. SaiPreethi, Statistical Analysis Of Medical Appointments Using Decision Tree, Conference on Science Technology Engineering and Mathematics (2019) 59-64. doi: 10.1109/ICONSTEM.2019.8918766
[Y3] Lior, R., Maimon, O.: Data mining with decision trees: theory and applications, vol. 81. World Scientific, 2 edn. (2014)
[Y4] Quinlan, J.R.: Induction of decision trees. Machine learning 1(1), 81–106 (1986)
[Y5] Jadhav SD, Channe HP. Efficient recommendation system using decision tree classifier and collaborative filtering. Int. Res. J. Eng. Technol. 2016;3:2113-8.
[Y6] Kavitha M, Gnaneswar G, Dinesh R, et al. Heart disease prediction using hybrid machine learning model[C]//2021 6th International Conference on Inventive Computation Technologies (ICICT). IEEE, 2021: 1329-1333.
[Y7] Chandrasekar P, Qian K, Shahriar H, et al. Improving the prediction accuracy of decision tree mining with data preprocessing[C]//2017 IEEE 41st Annual Computer Software and Applications Conference (COMPSAC). IEEE, 2017, 2: 481-484.


\subsubsection{Decision Tree}
\paragraph{Approach.}
Decision Tree is a method that is used widely due to the simplicity of its interpretation and application. Moreover, this classification method doesn't require a significant data preparation which highlights it among other methods. However, before the method application, we have searched the best combination of hyperparameters for it by using Grid Search with Cross Validation. Despite its 

\subsection{MLP}\label{subsubsec4}
As the most typical neural network, the multilayer perceptron (MLP) has a simple and regular structure and can fit any continuous function when the hidden layer design is sufficiently perfect. Our group used the MLPClassifier to clarify, The MLPClassifier class implements a Multilayer Perceptron (MLP) algorithm trained using Backpropagation.[4] The MLP is trained on two arrays: an array X of size (n_samples, n_features) that holds the training samples represented by floating-point feature vectors.[5] Through normalization, the features between different dimensions are numerically comparable to a certain extent, which can greatly improve the accuracy of the classifier.[6]

MLP parameter selection was carried out using GridSearchCV.[7]
The parameters are hidden_layer_sizes,alpha,learning_rate,max_iter. MLP commonly uses the backpropagation algorithm to find MLP parameters, it may be limited to local optimums. [8]

\textit{\textbf{reference}}
[1]Syrine Ben Driss, Mahmoud Soua, Rostom Kachouri, Mohamed Akil. A comparison study between MLP and Convolutional Neural Network models for character recognition. SPIE Conference on Real- Time Image and Video Processing, Apr 2017, Anaheim, CA, United States. 10.1117/12.2262589 . 
hal-01525504 
[2]Chow and Yee, 1991 M. Chow, S.O. Yee Application of neural networks to incipient fault detection in induction motors
[3]Journal of Neural Network Computing, 2 (3) (1991), pp. 26-32 Ilonen and Kamarainen, 2005 J. Ilonen, J.-K. Kamarainen Diagnosis tool for motor condition monitoring IEEE Transactions on Industry Applications, 41 (4) (2005), pp. 963-971
[4]B.C. Pak, Y.I. Cho Hydrodynamic and heat transfer study of dispersed fluids with submicron metallic oxide particles Exp. Heat Transf. Int. J., 11 (1998), pp. 151-170
[5]C. Nguyen, F. Desgranges, G. Roy, N. Galanis, T. Mare, S. Boucher, et al.Temperature and particle-size dependent viscosity data for water-based nanofluids–hysteresis phenomenon Int. J. Heat Fluid Flow, 28 (2007), pp. 1492-1506
[6]I.M. Krieger, T.J. Dougherty A mechanism for non-Newtonian flow in suspensions of rigid spheres Trans. Soc. Rheol., 3 (1959), pp. 137-152
[7]F. Yousefi, H. Karimi, M.M. Papari Modeling viscosity of nanofluids using diffusional neural networks J. Mol. Liq., 175 (2012), pp. 85-90
[8]R. Gilmont Liquid viscosity correlations for flowmeter calculations Chem. Eng. Prog., 98 (2002), pp. 36-41
\section{Random forest}
Due to the unique advantages of RF in handling small sample sizes, high-dimensional feature spaces and complex data structures [1], random Forest is a very common algorithm in machine learning and one of the most practical algorithms for bagging integration strategies. A random forest (RF) classifier uses a randomly selected subset of training samples and variables to generate multiple decision trees [2] and its natural incorporation of feature selection and interactions in the learning process [1]. A random forest is an ensemble of unpruned classification or regression trees created by using bootstrap samples of training data and random feature selection of the trees. Predictions are made by aggregating (majority voting or averaging) the predictions of the ensemble [3]. To meet the diversity requirement, random sampling of the dataset is required, which includes random sampling of samples and random sampling of features, with the aim of giving each tree a personality. It has been noted that random forests may have biased feature selection for individual trees [4]. Therefore, if secondary features are selected, the classification accuracy will be affected [5].
In order to obtain valid results, the parameters of the random forest had to be carefully tuned [6]. We chose GridSearchCV for parameter optimisation and tuning. The number of trees in a supervised learning Random Forest (RF) algorithm is set by the user [7]. More trees mean more overall capability, but modelling too many trees will result in some reduction in overall efficiency, and there is also a time cost to consider, so it is not the case that the greater the number of trees, the superior the performance of the forest. Doubling the number of trees is of no value, and there is no significant benefit beyond a threshold unless there is a huge computing environment [8]. So, we selected five representative values for the tree count screening. A variety of partitioning criteria exist in random forests. For example, Breiman et al [9] proposed the classical tree CART based on Gini impurity, which measures the probability that a randomly selected sample in the sample set is split wrongly. a smaller Gini index means that the probability that a selected sample in the set is split wrongly is smaller, which means that the set is more pure, and vice versa, the set is less pure. Quinlan [10] proposed a method based on information-theoretic view perspective using information gain as the segmentation criterion, which mainly considers the information that exists between the local nodes and the predicted output [11]. We filtered the segmentation criteria by GridSearchCV in two ways: 'gini', 'entropy'.

\section{Reference}
[1] Qi, Y., 2012. Random forest for bioinformatics. In Ensemble machine learning (pp. 307-323). Springer, Boston, MA.
[2] Belgiu, M. and Drăguţ, L., 2016. Random forest in remote sensing: A review of applications and future directions. ISPRS journal of photogrammetry and remote sensing, 114, pp.24-31.
[3] Svetnik, V., Liaw, A., Tong, C., Culberson, J.C., Sheridan, R.P. and Feuston, B.P., 2003. Random forest: a classification and regression tool for compound classification and QSAR modeling. Journal of chemical information and computer sciences, 43(6), pp.1947-1958.
[4] Strobl, C., Boulesteix, A.L., Zeileis, A. and Hothorn, T., 2007. Bias in random forest variable importance measures: Illustrations, sources and a solution. BMC bioinformatics, 8(1), pp.1-21.
[5] Paul, A., Mukherjee, D.P., Das, P., Gangopadhyay, A., Chintha, A.R. and Kundu, S., 2018. Improved random forest for classification. IEEE Transactions on Image Processing, 27(8), pp.4012-4024.
[6] Segal, M.R., 2004. Machine learning benchmarks and random forest regression.
[7] Probst, P. and Boulesteix, A.L., 2017. To tune or not to tune the number of trees in random forest. J. Mach. Learn. Res., 18(1), pp.6673-6690.
[8] Oshiro, T.M., Perez, P.S. and Baranauskas, J.A., 2012, July. How many trees in a random forest?. In International workshop on machine learning and data mining in pattern recognition (pp. 154-168). Springer, Berlin, Heidelberg.
[9] Timofeev, R., 2004. Classification and regression trees (CART) theory and applications. Humboldt University, Berlin, 54.
[10] Quinlan, J.R., 1986. Induction of decision trees. Machine learning, 1(1), pp.81-106.
[11] Yang, B.B., Gao, W. and Li, M., 2019, November. On the robust splitting criterion of random forest. In 2019 IEEE International Conference on Data Mining (ICDM) (pp. 1420-1425). IEEE.
\input{4.Result}